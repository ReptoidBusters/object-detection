\begin{frame}\frametitle{Постановка задачи}
    \begin{itemize}
        \item Задача: 
    \end{itemize}
\end{frame}
\begin{frame}\frametitle{Алгоритм}
    \begin{itemize}
        \item Определить ключевые точки на ключевом кадре и изображении
        \item Репроецировать точки ключевого кадра на объект.
        \item Сопоставить ключевые точки. Получить соотвествие 
            точек на изображении и точек на объекте.
        \item Определить положение объекта по соответствию 2D и 3D точек
    \end{itemize}
\end{frame}
\begin{frame}\frametitle{Определение ключевых точек}
    \begin{itemize}
        \item Точки, которые обладают некоторыми важными свойствами
        \item Устойчивость к деформациям, точность локализации
        \item Для данной задачи были выбраны
    \end{itemize}
\end{frame}
\begin{frame}\frametitle{Сопоставление ключевых точек}
    \begin{itemize}
        \item Вычисление дескрипторов (вектор 128)
        \item Сопоставление точек расстояние между дескрипторами которых минимально
    \end{itemize}
\end{frame}
\begin{frame}\frametitle{PnP}
    \begin{itemize}
        \item Задача PnP состоит в определении матрицы трансформации по соотвествию двухмерных и трёхмерных точек
        \item Проблема в том, что из-за не точности соотевствия ключевых точек есть точки-выбросы. 
            Для наборов с такими точками методы точного решения PnP дают очень плохие результаты
        \item Решение проблемы - RANSAC
    \end{itemize}
\end{frame}
\begin{frame}\frametitle{RANSAC}
    \begin{itemize}
        \item Random sample consensus
        \item Метод выбирает несколько случайных наборов точек размера пять. 
            Вероятность того, что ни одна из этих точек не является выбросом достаточна велика 
            если количество выбросов не более половины.
        \item По пяти точкам можно построить решение задачи PnP.
        \item С помощью полученной матрицы трансформации трёхмерные точки 
            проецируются на плоскость.
        \item Точки, лежащие в пределах допустимой погрешности считаются невыбросами
        \item Среди всех наборов выбирается тот, который дал наибольшее количество невыбросов
    \end{itemize}
\end{frame}
\begin{frame}\frametitle{Недостатки RANSAC}
    \begin{itemize}
        \item Погрешность
        \item Как следствие погрешности при недостаточном количестве набором RANSAC может 
            не найти ни одного набора, в котором есть невыбросы и не решить задачу.
        \item Но такое случается не часто :)
    \end{itemize}
\end{frame}



