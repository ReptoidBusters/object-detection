\begin{frame}\frametitle{Цель}
    \begin{itemize}
        \item Цель: определить положение объекта на изображении
    \end{itemize}
\end{frame}
\begin{frame}\frametitle{Задачи}
    \begin{itemize}
        \item Определить ключевые точки на ключевом кадре и изображении.
        \item Репроецировать точки ключевого кадра на объект.
        \item Сопоставить ключевые точки. Получить соотвествие 
            точек на изображении и точек на объекте.
        \item Определить положение объекта по соответствию 2D и 3D точек
    \end{itemize}
\end{frame}
\begin{frame}\frametitle{Определение ключевых точек}
    \begin{itemize}
        \item Ключевые точки - это точки, которые обладают некоторыми важными свойствами
        \item Важнейшие свойства: устойчивость к преобразованиям объекта, 
            точность локализации
        \item Для данной задачи были выбраны ключевые точки SIFT 
            (scale-invariant feature transform)
    \end{itemize}
\end{frame}
\begin{frame}\frametitle{Ключевые точки: пример}
    \begin{center}
        \includegraphics[height=6cm]{shevchenko_imgs/1.png}
    \end{center}
\end{frame}

\begin{frame}\frametitle{Сопоставление ключевых точек}
    \begin{itemize}
        \item Вычисление дескриптора, то есть объекта, который описывает
            ключевую точку.
        \item Дескриптором для ключевых точек SIFT является вектор из 
            градиентов в ближайших точках.
        \item Сопоставление точек расстояние между дескрипторами которых 
            минимально.
        \item Наивный метод.
    \end{itemize}
\end{frame}
\begin{frame}\frametitle{Сопоставление ключевых точки: пример}
    \begin{center}
        \includegraphics[height=6cm]{shevchenko_imgs/2.png}
    \end{center}
\end{frame}
\begin{frame}\frametitle{PnP}
    \begin{itemize}
        \item Задача PnP состоит в определении матрицы трансформации по соотвествию двухмерных и трёхмерных точек
        \item Проблема в том, что из-за неточности соответствия ключевых точек есть точки-выбросы. 
            Для наборов с такими точками методы точного решения PnP дают очень плохие результаты
        \item Решение проблемы - RANSAC
    \end{itemize}
\end{frame}
\begin{frame}\frametitle{RANSAC}
    \begin{itemize}
        \item Random sample consensus
        \item Метод выбирает несколько случайных наборов точек размера пять. 
            Вероятность того, что ни одна из этих точек не является выбросом достаточна велика 
            если количество выбросов не более половины.
        \item По пяти точкам можно построить решение задачи PnP.
        \item С помощью полученной матрицы трансформации трёхмерные точки 
            проецируются на плоскость.
        \item Точки-проекции, лежащие в пределах допустимой погрешности от 
            исходных точек изображения считаются невыбросами
        \item Среди всех наборов выбирается тот, который дал наибольшее 
            количество невыбросов.
    \end{itemize}
\end{frame}
\begin{frame}\frametitle{Финальный результат: ключевой кадр}
    \begin{center}
        \includegraphics[height=6cm]{shevchenko_imgs/3.png}
    \end{center}
\end{frame}
\begin{frame}\frametitle{Финальный результат: изображение}
    \begin{center}
        \includegraphics[height=6cm]{shevchenko_imgs/4.png}
    \end{center}
\end{frame}

\begin{frame}\frametitle{Результаты}
    \begin{itemize}
        \item Изучил различные типы ключевых точек и методы их выделения.
        \item Изучил метод RANSAC.
        \item Научился пользоваться библиотекой OpenCV в Python.
        \item Попробовал себе в командной разработке с использованием git.
    \end{itemize}
\end{frame}




